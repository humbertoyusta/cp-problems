\hypertarget{subtarea-1}{%
\subsubsection{Subtarea 1:}\label{subtarea-1}}

Analicemos el siguiente algoritmo, mientras \(A\) no sea igual a \(B\),
buscamos el primer índice en que difieren, llamémoslo \(x\), si
\(x \geq n-1\), entonces no hay solución, de lo contrario, si
\(A_i < B_i\), aplicamos una operación para sumarle \(1\) a \(A_i\),
restarle \(1\) a \(A_{i+1}\) y sumarle \(1\) a \(A_{i+2}\), y si
\(A_i > B_i\), aplicamos una operación para restarle \(1\) a \(A_i\),
sumarle \(1\) a \(A_{i+1}\) y restarle \(1\) a \(A_{i+2}\).

La implementación directa de lo anteriormente explicado es suficiente
para pasar la primera subtarea.

\hypertarget{subtarea-2}{%
\subsubsection{Subtarea 2:}\label{subtarea-2}}

El algoritmo de la subtarea anterior puede ser optimizado de la
siguiente forma, en orden ascendente, para cada \(i\) de \(1\) a \(n-2\)
si \(A_i < B_i\) hacemos \(B_i - A_i\) operaciones de sumarle \(1\) a
\(A_i\), restarle \(1\) a \(A_{i+1}\) y sumarle \(1\) a \(A_{i+2}\), si
\(A_i > B_i\) aplicamos \(A_i - B_i\) operaciones de restarle \(1\) a
\(A_i\), sumarle \(1\) a \(A_{i+1}\) y restarle \(1\) a \(A_{i+2}\). Si
al terminar este algoritmo, ambos arreglos son iguales la respuesta es
\(YES\), de lo contrario es imposible.

Complejidad \(O(n)\)

\hypertarget{subtarea-3}{%
\subsubsection{Subtarea 3:}\label{subtarea-3}}

Denotemos la cantidad de operaciones que se le hacen a la posición \(x\)
en el algoritmo anterior, como \(f(x)\), si \(f(x) > 0\) es que se le
hacen operaciones de suma, de lo contrario de resta.

Analicemos como se comporta \(f(x)\) durante el algoritmo, podemos decir
que:

\(f(x) = B_x - A_x + f(x-1) - f(x-2)\)

Y del arreglo \(A\) se puede llegar al arreglo \(B\) si y solo si
\(f(n-1)\) y \(f(n)\) son iguales a \(0\).

Ahora analicemos que ocurre con \(f\) cuando sumamos \(1\) a \(A_x\),
suponiendo que antes tuviéramos los \(f(i)\) de la siguiente manera:

\begin{verbatim}
0 0 0 0 0 0 0 0 0 0 0 0
\end{verbatim}

Ahora tendríamos lo siguiente:

\begin{verbatim}
-1 -1 0 1 1 0 -1 -1 0 1 1 0
\end{verbatim}

Se puede notar que \(f(y)\) aumentará o disminuirá lo mismo que \(f(x)\)
si \(y - x = 0\mod{6}\) o \(y - x = 1\mod{6}\), aumentará o disminuirá
en el opuesto que \(f(x)\) si \(y - x = 3\mod{6}\) o
\(y - x = 4\mod{6}\) y no cambiará si \(y - x = 2\mod{6}\) o
\(y - x = 5\mod{6}\).

Con lo siguiente podemos mantener \(f(n-1)\) y \(f(n)\) eficientemente
haciendo updates de sumar o restar \(x\) a una posición.

Complejidad \(O(n+q)\)

\hypertarget{subtarea-4}{%
\subsubsection{Subtarea 4}\label{subtarea-4}}

Podemos obseervar que si se le suma \(x\) a un rango de tamaño \(6\),
solo cambian las \(f\) de ellos \(6\), las de la derecha permanecen
igual, como se muestra:

\begin{verbatim}
-1 -1  0  1  1  0 -1 -1  0  1  1  0
   -1 -1  0  1  1  0 -1 -1  0  1  1
      -1 -1  0  1  1  0 -1 -1  0  1
         -1 -1  0  1  1  0 -1 -1  0
            -1 -1  0  1  1  0 -1 -1
               -1 -1  0  1  1  0 -1
                  -1 -1  0  1  1  0
\end{verbatim}

Entonces cualquier cambio de tamaño \(6\), que no contenga a \(n-1\) o a
\(n\) es insignificante, y podemos reducir un update de \((l,r,x)\) a
\((l+6,r,x)\) mientras se pueda, y cuando sea un update de tamaño menor
o igual a \(6\), lo resolvemos como en la subtarea anterior.

Complejidad \(O(n+q)\)
