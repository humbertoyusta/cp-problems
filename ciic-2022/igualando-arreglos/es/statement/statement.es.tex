\hypertarget{tiempo-luxedmite-1-segundo.}{%
\subparagraph{Tiempo límite: 1
segundo.}\label{tiempo-luxedmite-1-segundo.}}

\hypertarget{memoria-luxedmite-256mb.}{%
\subparagraph{Memoria límite: 256MB.}\label{memoria-luxedmite-256mb.}}

\hypertarget{igualando-arreglos}{%
\subsubsection{Igualando Arreglos}\label{igualando-arreglos}}

Dados dos arreglos \(A\) y \(B\), de \(n\) elementos cada uno, puede
realizar cualquiera de las siguientes operaciones cualquier cantidad de
veces:

\begin{itemize}
\item
  Seleccione un índice \(i\), tal que \(2 \leq i \leq n-1\), súmele
  \(1\) a \(A_{i-1}\), réstele \(1\) a \(A_{i}\) y súmele \(1\) a
  \(A_{i+1}\).
\item
  Seleccione un índice \(i\), tal que \(2 \leq i \leq n-1\), réstele
  \(1\) a \(A_{i-1}\), súmele \(1\) a \(A_{i}\) y réstele \(1\) a
  \(A_{i+1}\).
\end{itemize}

Además tendrá que procesar \(q\) updates de la forma \((l, r, x)\), que
consisten en sumarle \(x\) a \(A_{l}, A_{l+1}, ..., A_{r}\).

Después de cada update usted tendrá que decir si se puede convertir A en
B usando las operaciones dadas, note que las operaciones que usted haga
no se guardarán para próximos updates, pero los elementos a los que se
les sumó por el update si se guardan para próximos updates.

\hypertarget{restricciones}{%
\subsubsection{Restricciones:}\label{restricciones}}

\(3 \leq n \leq 2\cdot 10^5\)

\(0 \leq q \leq 2\cdot 10^5\)

\(1 \leq A_i, B_i \leq 10^6\) para cada \(i\) tal que
\(1 \leq i \leq n\)

\(1 \leq l_i \leq r_i \leq n\) y \(-10^6 \leq x_i \leq 10^6\) para cada
\(i\) tal que \(1 \leq i \leq q\)

\hypertarget{subtareas}{%
\subsubsection{Subtareas:}\label{subtareas}}

\begin{itemize}
\item
  Subtarea 1: \(1 \leq n \leq 100\), \(1 \leq A_i, B_i \leq 100\) para
  cada \(i\) tal que \(1 \leq i \leq n\), \(q = 0\)(no hay updates). (10
  puntos)
\item
  Subtarea 2: \(q = 0\)(no hay updates). (20 puntos)
\item
  Subtarea 3: para todos los updates, se cumple que \(l_i\) = \(r_i\).
  (40 puntos)
\item
  Subtarea 4: Sin restricciones adicionales. (30 puntos)
\end{itemize}

\hypertarget{formato-de-entrada}{%
\subsubsection{Formato de entrada:}\label{formato-de-entrada}}

Primero una línea con \(n\) y \(q\), dos enteros separados por un
espacio, el tamaño de los arreglos \(A\) y \(B\), y la cantidad de
updates respectivamente.

Luego una línea con \(n\) enteros \(A_1, A_2, ... , A_n\) separados por
un espacio.

Luego una línea con \(n\) enteros \(B_1, B_2, ... , B_n\) separados por
un espacio.

Luego \(q\) líneas, la \(i\)-ésima de ellas describiendo el \(i\)-ésimo
update, \(l_i, r_i, x_i\), separados por un espacio.

\hypertarget{formato-de-salida}{%
\subsubsection{Formato de salida:}\label{formato-de-salida}}

Debe imprimir \(q+1\) líneas, cada una debe contener la respuesta
después de cada update, la primera línea contiene la respuesta antes de
que se haga el primer update. Debe imprimir \(YES\) si se puede alcanzar
el arreglo \(B\) a partir de \(A\) usando las operaciones requeridas o
\(NO\) de lo contrario.

\hypertarget{ejemplo-de-entrada-1}{%
\paragraph{Ejemplo de entrada 1:}\label{ejemplo-de-entrada-1}}

\begin{verbatim}
5 0
1 2 3 2 2
2 3 5 1 5
\end{verbatim}

\hypertarget{ejemplo-de-salida-1}{%
\paragraph{Ejemplo de salida 1:}\label{ejemplo-de-salida-1}}

\begin{verbatim}
YES
\end{verbatim}

\hypertarget{ejemplo-de-entrada-2}{%
\paragraph{Ejemplo de entrada 2:}\label{ejemplo-de-entrada-2}}

\begin{verbatim}
12 12
17 23 2 35 32 8 24 4 28 18 4 28
20 19 8 31 34 5 26 3 23 20 2 28
2 6 8
4 4 -10
9 9 -1
11 12 7
10 10 -7
3 9 10
9 9 4
3 9 8
3 3 -8
3 5 4
10 10 -8
5 11 6
\end{verbatim}

\hypertarget{ejemplo-de-salida-2}{%
\paragraph{Ejemplo de salida 2:}\label{ejemplo-de-salida-2}}

\begin{verbatim}
NO
NO
NO
YES
NO
NO
NO
YES
NO
YES
NO
YES
NO
\end{verbatim}
