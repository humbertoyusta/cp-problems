\hypertarget{tiempo-luxedmite-4-segundos.}{%
\paragraph{Tiempo límite: 4
segundos.}\label{tiempo-luxedmite-4-segundos.}}

\hypertarget{memoria-luxedmite-512-mb.}{%
\paragraph{Memoria límite: 512 MB.}\label{memoria-luxedmite-512-mb.}}

\hypertarget{texto-del-problema}{%
\subsection{Texto Del Problema}\label{texto-del-problema}}

A C-3IC se le ha ocurrido una nueva hipótesis, él cree que para todo
entero positivo \(n\), y todo par de enteros \((i, j)\), tal que
\(1 \leq i \leq j \leq n\) se cumple que
\(\gcd {(i, n)} \cdot \gcd {(j , n)} = \gcd {(i \cdot j, n)}\).

C-3IC corrió hacia su PC para probar si su hipótesis era verdadera
probando muchos casos, rápidamente notó que para un entero \(n\) su
hipótesis era válida solo en algunos pares \((i, j)\), decepcionado por
no descubrir un nuevo teorema, quiere analizar las propiedades de los
pares de números que cumplen su hipótesis, para lo cual le pide que
resuelva la siguiente tarea:

Dado un entero positivo \(n\), cuente el número de pares de enteros
\((i, j)\) tales que \(1 \leq i \leq j \leq n\) y
\(\gcd {(i, n)} \cdot \gcd {(j , n)} = \gcd {(i \cdot j, n)}\).

Habrá \(T (1 \leq T \leq 10 ^ 6)\) casos de prueba, en cada uno se le
dará un entero \(n (1 \leq n \leq 4 \cdot 10 ^ 6)\).

\hypertarget{subtareas}{%
\subsubsection{Subtareas:}\label{subtareas}}

\begin{itemize}
\item
  Subtarea 1: \(1 \leq T \leq 40\), \(1 \leq n \leq 40\). ( 5 puntos )
\item
  Subtarea 2: \(1 \leq T \leq 100\), \(1 \leq n \leq 10 ^ 5\), se
  garantiza que el divisor más pequeño de \(n\) es mayor que \(100\).
  (19 puntos)
\item
  Subtarea 3: \(1 \leq T \leq 10^4\), \(1 \leq n \leq 10 ^ 5\), se
  garantiza que existe un primo \(p\), tal que \(p^k = n\) para un
  entero positivo \(k\). (22 puntos)
\item
  Subtarea 4: \(1 \leq T \leq 10 ^ 5\), \(1 \leq n \leq 10 ^ 5\). (25
  puntos)
\item
  Subtarea 5: \(1 \leq T \leq 10 ^ 6\), \(1 \leq n \leq 4 * 10 ^ 6\).
  (29 puntos)
\end{itemize}

\hypertarget{formato-de-entrada}{%
\subsubsection{Formato de Entrada:}\label{formato-de-entrada}}

La primera línea contendrá la cantidad de casos \(T\) a procesar.

A partir de ahí, seguirán \(T\) líneas, cada una con un entero \(n\).

\hypertarget{formato-de-salida}{%
\subsubsection{Formato de Salida:}\label{formato-de-salida}}

Debe imprimir \(T\) líneas, cada una con un entero, la solución del
problema para cada caso.

\hypertarget{ejemplo-de-entrada}{%
\subsubsection{Ejemplo de entrada:}\label{ejemplo-de-entrada}}

\begin{verbatim}
6
1
2
3
4
5
10000
\end{verbatim}

\hypertarget{ejemplo-de-salida}{%
\subsubsection{Ejemplo de salida:}\label{ejemplo-de-salida}}

\begin{verbatim}
1
2
5
8
14
46047940
\end{verbatim}
