\hypertarget{subtask-1}{%
\subsection{Subtask 1:}\label{subtask-1}}

Note that you can go through all the pairs with brute force and ask if
they are valid, complexity \(O (T \cdot n^2 \cdot \log {n})\),
\((T \cdot n^3)\) or a little more depending on how the \(gcd\) is
calculated.

\hypertarget{subtask-2}{%
\subsection{Subtask 2:}\label{subtask-2}}

First notice that:

\begin{itemize}
\tightlist
\item
  A \((i, j)\) pair is valid if and only if
  \(gcd (i, n) \cdot gcd (j, n)\) is a divisor of \(n\).
\end{itemize}

Note that since the lowest prime is greater than \(100\), and the
numbers are up to \(10 ^ 5\), \(n\) can only be:

\begin{enumerate}
\def\labelenumi{\arabic{enumi}.}
\item
  A prime, in which case the answer is
  \(\frac{n \cdot {(n + 1)}}{2} -1\), that is, all pairs \((i, j)\)
  minus \((n , n)\).
\item
  The multiplication of two primes \(p\) and \(q\), in which case we
  must count all the pairs, and then discount the pairs \((i, j)\) such
  that they meet one of the following conditions:

  \begin{itemize}
  \item
    \(gcd (i, n) = p\) and \(gcd (j, n) = p\), these pairs can be
    counted as \((\frac {n} {p} -1) \cdot (\frac {n} {p} -1)\).
  \item
    \(gcd (i, n) = q\) and \(gcd (j, n) = q\), we can count them in a
    similar way to the previous ones.
  \item
    \(gcd (i, n) = p \cdot q\) and \(gcd (j, n) = p\), these can be
    counted as \(\frac {n} {p}\).
  \item
    \(gcd (i, n) = p \cdot q\) and \(gcd (j, n) = q\), we can count
    these as \(\frac {n} {q}\).
  \end{itemize}
\item
  The square of a prime \(p\), here we can count all the pairs and
  subtstract those that \(gcd (i, n) \neq 1\) and \(gcd (j, n) = n\),
  which we can do in a similar way to the previous case, or as
  \(n - \phi {n}\).
\end{enumerate}

Complexity \(O (T \cdot \sqrt {n})\) or maybe less depending on the
factorization.

\hypertarget{subtask-3}{%
\subsection{Subtask 3}\label{subtask-3}}

In this subtask \(n\) is the power of a prime, using that fact we can
count the number of elements with \(gcd (i, n) = p ^ x\) as
\(\frac{n}{p^x} - \frac{n}{p^{x+1}}\).

We can go through the pairs \((a, b)\) such that \(p ^ {a + b}\) does
not divide \(n\), and for each of them count the number of pairs that
have \(gcd (i, n) = p ^ a\) ans \(gcd (j, n) = p ^ b\) as explained in
the previous text, so we count the bad pairs and subtract them from the
total.

Complexity \(O(T \cdot \sqrt {n} + T \cdot \log^2 n)\), maybe less
depending on the factorization.

\hypertarget{subtask-4}{%
\subsection{Subtask 4}\label{subtask-4}}

It can be shown that:

\begin{itemize}
\tightlist
\item
  The quantity of numbers \(x\) such that \(1 \leq x \leq n\) and
  \(gcd (x, n) = d\) is equal to \(\phi {\frac {n} {d}}\) (here
  \(\phi {m}\) is the quantity of numbers from \(1\) to \(m\) which are
  coprime with \(m\)), if and only if \(d\) is a divisor of \(n\),
  otherwise it is \(0\). This can be thought as if we divided each
  element by \(d\), those that are not divisible doesn't matter, we are
  interested in those that are equal to \(1\), which are coprime with
  \(\frac{n}{d}\).
\end{itemize}

Let us denote as \(F (x, y)\) the number of pairs \((i, j)\) such that:

\begin{itemize}
\tightlist
\item
  \(1 \leq i \leq n\)
\item
  \(1 \leq j \leq n\)
\item
  \(gcd (i, n) = x\)
\item
  \(gcd (j, n) = y\)
\item
  \(x \neq y\).
\item
  \(x \cdot y\) is a divisor of \(n\).
\end{itemize}

\(F (x, y)\) can be calculated as
\(\phi {\frac {n} {x}} \cdot \phi {\frac {n} {y}}\).

Furthermore, let's denote \(E (x)\) as the number of pairs \((i, j)\)
such that:

\begin{itemize}
\tightlist
\item
  \(1 \leq i \leq n\)
\item
  \(1 \leq j \leq n\)
\item
  \(gcd (i, n) = x\)
\item
  \(gcd (j, n) = x\)
\item
  \(x ^ 2\) is a divisor of \(n\).
\end{itemize}

\(E (x)\) can be calculated as
\(\frac {\phi {\frac {n} {x}} \cdot (\phi {\frac {n} {x} + 1)}} {2}\)

Now \(\sum_{x, y \in d (n)} F (x, y)\) (here \(d (n)\) is the list that
contains all the divisors of \(n\)) would count all the pairs in the
answer twice, except for those that have the same gcd, which would be
\(\sum_{x \in d (n)} E (x)\), let's denote \(f (n)\) as
\(\sum_{x, y \in d (n)} F (x, y)\) and \(g (n)\) as
\(\sum_ {x \in d (n)} E (x)\), the answer to the problem would be
\(\frac {f (n) + 2 \cdot g (n)} {2}\).

In this subtask you can calculate the solutions simply by checking each
pair of divisors of each number and calculating \(F (x, y)\) and
\(E (x)\) as explained above.

You can save the solutions, since the sum of the number of divisors of
each number up to \(n\) is \((n \cdot \log n)\) and the number of
divisors of a number \(n\) is \(O (n ^ {\frac {1} {3}})\).

Complexity: \(O (\sum_{i = 1}^{n} d(i) ^ 2)\) with a low constraint.
(here \(d (i)\) is the number of divisors of the number \(i\))

\hypertarget{subtask-5}{%
\subsection{Subtask 5}\label{subtask-5}}

It can be shown that the functions \(g\) and \(h\) are multiplicative,
compute them for the powers of primes similar to subtask 3, and for the
rest use the fact that they are multiplicative.

The fact that \(h\) is multiplicative is not very important, since it
can be computed in \(O (n \log {n})\) without any trouble without
noticing that is multiplicative, but the fact that \(g\) is
multiplicative is important.

It can be shown that \(g\) is multiplicative, due to the fact that if we
separate the problem into powers of primes, a pair will be valid in the
multiplication of the powers of primes if and only if it is valid in
each of these powers of primes.

Complexity: \(O (n \cdot \log {n})\) or \(O (n)\) depending on the
implementation.
