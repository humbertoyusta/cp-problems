\hypertarget{subtarea-1}{%
\subsection{Subtarea 1:}\label{subtarea-1}}

Note que se puede recorrer todos los pares con fuerza bruta y preguntar
si son válidos, complejidad \(O(T\cdot n^2 \cdot log{n})\),
\((T\cdot n^3)\) o un poco más dependiendo como se calcule el \(gcd\).

\hypertarget{subtarea-2}{%
\subsection{Subtarea 2:}\label{subtarea-2}}

Primero note que:

\begin{itemize}
\tightlist
\item
  Un par \((i, j)\) es válido si y solo si \(gcd(i, n) * gcd(j, n)\) es
  un divisor de \(n\).
\end{itemize}

Note que como el menor primo es mayor que \(100\), y los números son
hasta \(10^5\), \(n\) solo puede ser:

\begin{enumerate}
\def\labelenumi{\arabic{enumi}.}
\item
  Un primo, en cuyo caso la respuesta es \(\frac{n\cdot{(n+1)}}{2}-1\),
  o sea, todos los pares \((i, j)\) menos \((n, n)\).
\item
  La multiplicación de dos primos \(p\) y \(q\), en cuyo caso debemos
  contar todos los pares, y luego descontar los pares \((i, j)\) tal que
  cumplan una de las siguientes condiciones:

  \begin{itemize}
  \item
    \(gcd(i,n) = p\) y \(gcd(j,n) = p\), estos pares la podemos contar
    como \((\frac{n}{p}-1) \cdot (\frac{n}{p}-1)\).
  \item
    \(gcd(i,n) = q\) y \(gcd(j,n) = q\), los podemos contar de manera
    similar a los anteriores.
  \item
    \(gcd(i,n) = p \cdot q\) y \(gcd(j,n) = p\), estos los podemos
    contar como \(\frac{n}{p}\).
  \item
    \(gcd(i,n) = p \cdot q\) y \(gcd(j,n) = q\), estos los podemos
    contar como \(\frac{n}{q}\).
  \end{itemize}
\item
  El cuadrado de un primo \(p\), aquí podemos contar todos los pares y
  descontar los que \(gcd(i,n) \neq 1\) y \(gcd(j,n) = n\), que podemos
  hacerlo de manera similar al caso anterior, o como \(n - \phi{n}\).
\end{enumerate}

Complejidad \(O(T \cdot \sqrt{n})\) o quizá menos dependiendo de la
factorización.

\hypertarget{subtarea-3}{%
\subsection{Subtarea 3}\label{subtarea-3}}

Como en esta subtarea \(n\) es la potencia de un primo podemos contar la
cantidad de elementos con \(gcd(i, n) = p^x\) como
\(\frac{n}{p^x} - \frac{n}{p^{x+1}}\).

Podemos recorrer los pares \((a, b)\) tal que \(p^{a+b}\) no divide a
\(n\), y para cada uno de ellos contar la cantidad de pares que tienen
\(gcd(i,n) = p^a\) y \(gcd(j,n) = p^b\) como se explica en el texto
anterior, así contamos los pares malos y se los restamos al total.

Complejidad \(O(T \cdot \sqrt {n} + T \cdot \log^2 n)\), quizá menos
dependiendo de la factorización.

\hypertarget{subtarea-4}{%
\subsection{Subtarea 4}\label{subtarea-4}}

Note que:

\begin{itemize}
\tightlist
\item
  La cantidad de números \(x\) tales que \(1 \leq x \leq n\) y
  \(gcd(x,n) = d\) es igual a \(\phi{\frac{n}{d}}\) (donde \(\phi{m}\)
  es la cantidad de números de \(1\) a \(m\) coprimos con \(m\)), si y
  solo si \(d\) es un divisor de \(n\), de lo contrario es \(0\). Esto
  se puede intuir como si dividieramos cada elemento por \(d\), los que
  no sean divisibles no influyen de los restantes nos interesan los que
  son iguales a \(1\), o sea, los que son coprimos con \(\frac{n}{d}\).
\end{itemize}

Denotemos como \(F(x, y)\) a la cantidad de pares \((i, j)\) tales que:

\begin{itemize}
\tightlist
\item
  \(1 \leq i \leq n\)
\item
  \(1 \leq j \leq n\)
\item
  \(gcd(i, n) = x\)
\item
  \(gcd(j, n) = y\)
\item
  \(x \neq y\).
\item
  \(x * y\) es un divisor de \(n\).
\end{itemize}

\(F(x, y)\) puede ser calculado como
\(\phi{\frac{n}{x}} \cdot \phi{\frac{n}{y}}\).

Además denotemos \(E(x)\) como la cantidad de pares \((i, j)\) tales
que:

\begin{itemize}
\tightlist
\item
  \(1 \leq i \leq n\)
\item
  \(1 \leq j \leq n\)
\item
  \(gcd(i, n) = x\)
\item
  \(gcd(j, n) = x\)
\item
  \(x^2\) es un divisor de \(n\).
\end{itemize}

\(E(x)\) puede ser calculado como
\(\frac{\phi{\frac{n}{x}} \cdot (\phi{\frac{n}{x} + 1)}}{2}\)

Ahora \(\sum_{x,y \in d(n)} F(x,y)\) (donde \(d(n)\) es la lista que
contiene a todos los divisores de \(n\)) contaría todos los pares en la
respuesta dos veces, excepto los que tienen igual gcd, los cuales serían
\(\sum_{x \in d(n)} E(x)\), denotemos \(f(n)\) como
\(\sum_{x,y \in d(n)} F(x,y)\) y \(g(n)\) como
\(\sum_{x \in d(n)} E(x)\), la respuesta del problema sería
\(\frac{f(n) + 2\cdot g(n)}{2}\).

En esta subtarea se puede calcular las soluciones simplemente chequeando
cada par de divisores de cada número y calculando \(F(x, y)\) y \(E(x)\)
como se explica anteriormente.

Puede guardar las soluciones, ya que la suma de la cantidad de divisores
de cada número hasta \(n\) es \(O(n\cdot \log n)\) y la cantidad de
divisores de un número \(n\) es \(O(n^{\frac{1}{3}})\).

Complejidad: \(O(\sum_{i=1}^{n} d(i)^2)\) con baja constante. (donde
\(d(i)\) es la cantidad de divisores del número \(i\))

\hypertarget{subtarea-5}{%
\subsection{Subtarea 5}\label{subtarea-5}}

Se puede notar que las funciones \(f\) y \(g\) son multiplicativas,
calcular para las potencias de primos similar a la subtarea 3, y para el
resto usar el hecho de que son multiplicativas.

El hecho de que \(g\) sea multiplicativa no es muy importante, ya que se
puede calcular en \(O(n\log{n})\) sin dificultad, en el caso de \(f\) si
es importante.

Se puede intuir que \(f\) es multiplicativa, por el hecho de que si
separamos el problema en potencias de primos, un par será valido en la
multiplicación de las potencias de primos si y solo si es válido en cada
una de estas potencias de primos.

Complejidad: \(O(n\cdot \log{n})\) o \(O(n)\) dependiendo de la
implementación.
